\documentclass[]{article}
\usepackage{lmodern}
\usepackage{amssymb,amsmath}
\usepackage{ifxetex,ifluatex}
\usepackage{fixltx2e} % provides \textsubscript
\ifnum 0\ifxetex 1\fi\ifluatex 1\fi=0 % if pdftex
  \usepackage[T1]{fontenc}
  \usepackage[utf8]{inputenc}
\else % if luatex or xelatex
  \ifxetex
    \usepackage{mathspec}
  \else
    \usepackage{fontspec}
  \fi
  \defaultfontfeatures{Ligatures=TeX,Scale=MatchLowercase}
\fi
% use upquote if available, for straight quotes in verbatim environments
\IfFileExists{upquote.sty}{\usepackage{upquote}}{}
% use microtype if available
\IfFileExists{microtype.sty}{%
\usepackage{microtype}
\UseMicrotypeSet[protrusion]{basicmath} % disable protrusion for tt fonts
}{}
\usepackage{hyperref}
\hypersetup{unicode=true,
            pdfborder={0 0 0},
            breaklinks=true}
\urlstyle{same}  % don't use monospace font for urls
\IfFileExists{parskip.sty}{%
\usepackage{parskip}
}{% else
\setlength{\parindent}{0pt}
\setlength{\parskip}{6pt plus 2pt minus 1pt}
}
\setlength{\emergencystretch}{3em}  % prevent overfull lines
\providecommand{\tightlist}{%
  \setlength{\itemsep}{0pt}\setlength{\parskip}{0pt}}
\setcounter{secnumdepth}{0}
% Redefines (sub)paragraphs to behave more like sections
\ifx\paragraph\undefined\else
\let\oldparagraph\paragraph
\renewcommand{\paragraph}[1]{\oldparagraph{#1}\mbox{}}
\fi
\ifx\subparagraph\undefined\else
\let\oldsubparagraph\subparagraph
\renewcommand{\subparagraph}[1]{\oldsubparagraph{#1}\mbox{}}
\fi

\date{}

\begin{document}

\subsection{Title}\label{title}

Phase-Change Materials for Group-IV Electro-Optical Switching and
Modulation

\subsection{Authors}\label{authors}

\begin{itemize}
\tightlist
\item
  Richard Soref
\end{itemize}

\subsection{Abstract}\label{abstract}

2 x 2 and 1 x 4 optical routing switches

The electrically actuated optical layer is a thin film of
\(\mathrm{Ge_2 Sb_2 Te_5}\) or GeTe or GeSe whose thickness is in the 10
to 100-nm range

This film was embedded at the centerline of an SOI, SON, or GeOI channel
waveguide structure for \emph{MZI} and \emph{directional-coupler}
switching

Alternatively, to switch free-space light beams, \textbf{Ge prisms}
sandwiching the film were employed.

\subsection{Highlight}\label{highlight}

The PCM index is \(\mathrm{n_{am} + i\,k_{am}}\) and
\(\mathrm{n_{cr} + i\,k_{cr}}\) in \emph{am} and \emph{cr},
respectively. Hence the phase change gives \textbf{electrorefraction}
(ER) of \(\mathrm{\Delta n = n_{cr} -n_{am}}\), and
\textbf{electro-absorption} (EA) of
\(\mathrm{\Delta k = k_{cr} - k_{am}}\). These EA and ER are stronger
than those seen in \emph{any prior} EO effect. For an inline, waveguided
1 x 1 modulator device, EA can be optimized to provide deep intensity
modulation over a few microns of length. However, for digital
modulation, the modulator speed will be limited to
\textbf{\textasciitilde{}10 Mb/s}. Therefore, it is preferable to
configure this 1 x 1 modulator as a \textbf{low-speed VOA}.

My colleagues and I have investigated \(\mathrm{Ge_2 Sb_2 Te_5}\) (GST),
GeTe and GeSe PCMs and their index properties are as follows:\\
GST (1.55 $\mathrm{\mu m}$)
\(\mathrm{n_{am} + i\,k_{am} = 4.60 + i\,0.12,\;n_{cr} + i\,k_{cr} = 7.45 + i\,1.49}\) ;\\
GeTe (1.55 $\mathrm{\mu m}$)
\(\mathrm{n_{am} + i\,k_{am} = 4.22 + i\,0.12,\;n_{cr} + i\,k_{cr} = 6.81 +i\,0.30}\) ;\\
GeSe (1.55 $\mathrm{\mu m}$)
\(\mathrm{n_{am} + i\,k_{am} = 2.40 + i\,0.00006,\;n_{cr} + i\,k_{cr} = 2.97 + i\,0.00006}\) ;\\
GST (2.1 $\mathrm{\mu m}$)
\(\mathrm{n_{am} + i\,k_{am} = 4.10 + i\,0.006,\;n_{cr} + i\,k{cr} = 6.85 + i\,0.50}\) .

Clearly, large EA is available for VOAs and SLMs. However, in the switch
category, we do not want EA. Low IL and low CT in the 2 x 2 are obtained
only when \textbf{ER \textgreater{}\textgreater{} EA and
\(\mathbf{k_{am}}\) \textless{} 0.007}.

\newpage
The strategy of minimizing both \(\mathrm{k_{am}\,and\;k_{cr}}\) comes
down to a choice of PCM and operation-wavelength. Each PCM has \emph{am}
and \emph{cr} bandgap-wavelengths. \textbf{For low k's the operation
wavelength should be longer than both bandgap wavelengths}.

\paragraph{RESULTS OF SIMULATIONS ON EO
SWITCHES}\label{results-of-simulations-on-eo-switches}

We investigated two 2 x 2 switches: the MZI and the three-waveguide (3W)
directional coupler. Here are results at 2100-nm for 10-nm of GST
embedded midlevel in 840nm x 420nm cross-section SOI nanowires: the MZI
with TM polarization and 38 Pm-length active region gives cross-state
IL(0.5dB) with CT\\(-15dB) and bar-state IL(1.1dB) with CT
(-16dB); the 3W with TE polarization and 568 Pm active length in the central waveguide
gives second-bar IL(0.8dB) with CT(-16dB) and first-cross IL(1.0dB) with
CT(-18dB). Turning to 1550 nm where a 20-nm layer of GeSe was simulated
in SOI waveguides of 620nm x 310nm cross-section, we found for the TE
polarization that an active length of 19 Pm gave 2 x 2 MZI switching
while an interaction length of 115 Pm sufficed for the 3W switch.

In one arm of the MZI, there is end-fire coupling between the
PCM-embedded segment and the uniform bulk silicon nanowires at the
segment's input and output. These two interfaces lead to an added IL for
the MZI which is about 1.2 dB for TE and 0.17 dB for TM in the \emph{cr}
state.

\subsection{Related work}\label{related-work}

\textbf{Future work:} Looking to the future, it is likely that further $\mathrm{k_{cr}}$
reduction at 1.55  would be attained with the PCM \(\mathrm{Si_1 Sb_2 Te_4}\)
whose \(\mathrm{E_g(am)}\) = 0.80 eV and \(\mathrm{E_g(cr)}\) = 0.61 eV
are slightly larger than those of GeSe. Silicon-containing PCMs like
\(\mathrm{Si_1 Sb_2 Te_4}\) might offer gaps even wider.

\paragraph{References}\label{references}

\begin{itemize}
\tightlist
\item
  Phase-change materials for electro-optical switching in the near- and
  mid-infrared
\end{itemize}

\end{document}
